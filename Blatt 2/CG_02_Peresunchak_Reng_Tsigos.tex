\documentclass[11pt]{article}
\usepackage[ngerman]{babel}
\usepackage[a4paper, left=2.5cm, right=2.5cm, top=3.5cm]{geometry}
\usepackage{amsmath}
\usepackage{nccmath}
\usepackage{amssymb}
\usepackage[pdftex]{graphicx}
\usepackage{listings}
\usepackage{booktabs}
\usepackage{tikz}
\usepackage{enumitem}
\usepackage{fancyhdr}
\usepackage{geometry}
\usepackage{mathtools}
\usepackage{lipsum}
\usepackage{amsthm}
\usepackage{amsfonts}
\usepackage{geometry}
\usepackage[dvipsnames]{xcolor}
\usepackage{enumitem}
\usepackage{mathtools}
\usepackage{fancyhdr}
\usepackage{listings}
\usepackage{tikz}
\usepackage[utf8]{inputenc}
\newcommand{\bbN}{\mathbbm{N}}
\newcommand{\bbZ}{\mathbbm{Z}}
\newcommand{\bbQ}{\mathbbm{Q}}
\newcommand{\bbR}{\mathbbm{R}}
\newcommand{\bbC}{\mathbbm{C}}
\newcommand{\bbI}{\mathbbm{I}}
\newcommand{\bigO}{\mathcal{O}}
\newcommand{\bigT}{\mathcal{T}}
\newcommand\numberthis{\addtocounter{equation}{1}\tag{\theequation}}
\newcommand\abv[2]{\stackrel{\mathclap{\normalfont\mbox\tiny{#2}}}{#1}}
\newcommand\norm[1]{\left\lVert{#1}\right\rVert}
\lstset{basicstyle=\ttfamily, mathescape}
\geometry{top=2.5cm, bottom=2.5cm, left=2.5cm, right=2.5cm}
\setlength{\parindent}{0 pt}
\hbadness=99999 % No fucking underfull hbox warning
\hfuzz=9999pt % No fucking overfull hbox warning
\pagestyle{fancy}
\tikzstyle{textbox} = [draw, rounded corners, minimum height=2em, minimum width=4em]
\tikzstyle{arrow} = [->, thick]
\setlength{\headheight}{42.91258pt}
\lstdefinestyle{pretty}{
  backgroundcolor=\color{gray!10},
  frame=single,
  rulecolor=\color{gray!60},
  basicstyle=\ttfamily\small,
  keywordstyle=\color{blue}\bfseries,
  commentstyle=\color{green!60!black}\itshape,
  stringstyle=\color{red!70!black},
  numberstyle=\tiny\color{gray},
  numbers=left,
  stepnumber=1,
  numbersep=8pt,
  tabsize=2,
  breaklines=true,
  breakatwhitespace=true,
  showstringspaces=false,
  captionpos=b
}

\lstset{style=pretty}
\lstset{escapeinside={(*@}{@*)}}

\lhead{Computergraphik\\
Wintersemester 2025/26\\
\today}
\chead{\huge\textbf{Übungsblatt 2}}
\rhead{Maximilian Peresunchak 3232875\\
Nico Reng 3731402\\
Viorel Tsigos 3720183}
\begin{document}
\section*{Aufgaben 1 - 4:}
Siehe Code

\section*{Aufgabe 5:}
\begin{enumerate}
  \item Durch die sehr geringe Helligkeit in der Nacht, bekommen die Zapfen im Auge die für das Farbsehen verantwortlich sind, zu wenig Licht ab. D.h. durch zu wenig Licht können die Zapfen nicht mehr richtig arbeiten und die Farbwahrnehmung ist stark eingeschränkt. Die Stäbchen hingegen, brauchen weniger Licht und sind eher lichtempflindlich. Die Stäbchen sind dafür verantwortlich, für die Hell-Dunkel-Wahrnehmung. Also sieht man in der Nacht eher Grautöne und weniger Farben, deshalb sind nachts alle Katzen grau.
  \item Dadurch dass die helle Cyan-ähnlihce Schriftfarbe auf einem weißen Hintergrund sehr wenig Kontrast aufweist, ist es für das Auge schwer die Schrift zu erkennen. Dadurch dass die beiden Farben eine sehr hohe Luminaz haben, verschwimmt deshalb die Schrift mit dem Hintergrund. Wir wissen die Luminazgleichung lautet: 
  \begin{align*}
    Y = 0.2126 \cdot R + 0.7152 \cdot G + 0.0722 \cdot B
  \end{align*}
  Wir normalisieren zuerst die RGB-Werte, also von 0-255 auf 0-1: \\ \\
  \underline{Cyan-ähnliche Schrift:} 
  \begin{align*}
    R_c = 100/255 \approx 0.392 \\
    G_c = 255/255 \approx 0.941 \\
    B_c = 255/255 = 1 \\
    \end{align*}

  \underline{Weißer Hintergrund:} 
  \begin{align*}
    R_w = 255/255 = 1 \\
    G_w = 255/255 = 1 \\
    B_w = 255/255 = 1 \\
  \end{align*}
  Jetzt berechnen wir die Luminaz für die Cyan-ähnliche Schrift und für den weißen Hintergrund:
  \begin{align*}
    Y_c &= 0.2126 \cdot 0.392 + 0.7152 \cdot 0.941 + 0.0722 \cdot 1 \\
    &\approx 0.0833 + 0.6730 + 0.0722 \\
    &\approx 0.8285
  \end{align*}
  \begin{align*}
    Y_w &= 0.2126 \cdot 1 + 0.7152 \cdot 1 + 0.0722 \cdot 1 \\
    &\approx 0.2126 + 0.7152 + 0.0722 \\
    &\approx 1
  \end{align*}
  Wir können jetzt noch den Kontrast zwischen den beiden Farben berechnen:
  \begin{align*}
    \text{Kontrast} = \frac{Y_w }{Y_c} = \frac{1}{0.8285} \approx 1.149 : 1 \approx 1.15 : 1
  \end{align*}
  Dadurch dass die beiden Farben einen sehr geringen Kontrast von ca. 1.15:1 haben, ist es für das Auge schwer die Schrift zu erkennen. 
  \item Das menschliche Auge hat drei verschiedene Typen von Zapfen: S-, M- und L-Zapfen\\
  Die Wahrnehmung der Farben basiert auf verschiedenen Kanälen. Luminanzkanal (L + M), Rot-Grün-Kanal (L - M) und Blau-Gelb-Kanal (S - (L + M)). Deshalb ist es nicht möglich dass eine Rot-Blau-Schwäche exisitiert. 
  \begin{figure}[htbp]
    \centering
    \includegraphics[width=0.6\textwidth]{images/Farbenblindheit.jpg}
    \caption{Quelle: 01CG25\_Farbe\_Displays - Seite 44}
  \end{figure}
  \item Metamere sind Lichtspektren. Es sind eigentlich verschiedene Wellenlängen die aber vom menschlichen Auge als die gleiche Farbe wahrgenommen werden. Es sind also eigentlich verschiedene Farbreize die aber für den Menschen gleich wirken. Wenn nämlich zwei unterschiedliche Spektren die gleiche Reizstärke in allen drei Zapftypen die wir vorher genannt haben auslösen, werden diese als die gleiche Farbe wahrgenommen. 
  Folgende Ursachen können Metamere hervorrufen:
  \begin{enumerate}
    \item[1.] \underline{Beleuchtungswechsel:} \\
    Zwei Objekte können die gleiche Farbe unter einer Beleuchtung haben. Ändert sich jedoch der Lichteinfall (z.B. Tageslicht zu Kunstlicht), so können die zuvor gleich aussehenden Objekte unterschiedliche Farben aufweisen. Das ist wegen den Reflexionsspektren der Fall. Diese Metamerie bezeichnet man auch als Beleuchtungsmetamerie. \\

    \item[2.] \underline{Winkelveränderung:} \\
    Wieder zwei Objekte die unter einem bestimmten Blickwinkel die gleiche Farbe haben. Verändert sich der Winkel zwischen der Oberfläche und dem Auge, so kann es sein dass der Farbeindruck sich ändert. Das liegt dann an den Oberflächeneffekten. Das ist dann eine Geometriemetamerie.
    Wieder zwei Objekte die unter einem bestimmten Blickwinkel die gleiche Farbe haben. Verändert sich der Winkel zwischen der Oberfläche und dem Auge, so kann es sein dass der Farbeindruck sich ändert. Das liegt dann an den Oberflächeneffekten. Das ist dann eine Geometriemetamerie. 
  \end{enumerate}
\end{enumerate}
\end{document}